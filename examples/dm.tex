\hypertarget{premium-allocation-and-default-priority-rules}{%
\section{Premium Allocation and Default Priority
Rules}\label{premium-allocation-and-default-priority-rules}}

Nov 25, 2018

Derived from MM\_2RM

\hypertarget{three-important-differentials-homogeneous-case}{%
\subsection{Three important differentials (homogeneous
case)}\label{three-important-differentials-homogeneous-case}}

From Tasche and other references. Let \(X_u=X=u_1X_1 + u_2X_2\) be a
homogeneous portfolio.

The deriviative of the survival function \(S_u(t)=\text{Pr}(X_u>t)\) is
\[
\nabla_u S(t) = \text{E}[X_i 1_{\{X=t\}}] = \text{E}[X_i \mid X=t] f_u(t)
\] where \(f_u\) is the sensity of \(X_u\).

The deriviative of value at risk \(Q(u)\) for \(X_u\) at a fixed
threshold \(p\) is \[
\nabla_u Q(u) = E[X_i\mid \sum u_ix_i = Q(u,p)]
\]

The deriviative of tail value at risk \(T\) is \[
\nabla_u T(u) = E[X_i\mid \sum u_ix_i>Q(u,p)]
\]

\hypertarget{method-a}{%
\subsection{Method A}\label{method-a}}

For M\textbf{A}jor :-)

Homogeneous portfolio \(X_u=X=u_1X_1 + u_2X_2\).

Regulatory capital measure \(\alpha\), typically VaR or TVaR.
\textbf{Assume \(\alpha=\)TVaR} at a threshold \(p\) close to 1. Thus
\(\alpha(X_u) = \text{E}[X\mid X > F_u^{-1}(p)]\). Total assets
\(\alpha\) have a natural co-measure allocation given by
\(\alpha_i(X_u)= \text{E}[u_iX_i \mid X_u > F_u^{-1}(p)] = u_i \partial T/\partial u_i\).

Distortion \(g\) determines market pricing for risk transfer, via
\(E_g(X) = \int g(S_X(t))dt\).

Price is determined by using \(g\) to price
\(X\wedge \alpha(X):=\text{min}(X, \alpha(X))\).

The resulting pricing functional \(\pi\) can be written in several ways.
Let \(F_u^{-1}\) be the quantile function (i.e.~value at risk) of
\(X_u\). Then \[
\begin{align}
\pi(X_u) &= \int_0^{\alpha(X)} g(S_u(t))dt \\
&= \text{E}[(X_u\wedge \alpha(X_u)) g'(S_u(X)] \\
&= \int_0^1 (F_u^{-1}(p)\wedge \alpha(X_u)) g'(1-p)dp. \\
\end{align}
\]

We both agree that, subject to certain continuity and differentiability
assumptions, \[
\begin{align}
\frac{\partial\pi}{\partial u_i}(u) &= \text{E}\left[\left(X_i 1_{\{X\le \alpha\}} + E[X_i\mid X >  F_u^{-1}(p)]1_{\{X > \alpha\}}\right) g'(S_u(X_u) \right] \\
&= \text{E}[ X_i g'(S(X)) \mid X \le \alpha](1-g(S(\alpha)))+ \text{E}[X_i \mid X > F_u^{-1}(p)]g(S(\alpha)).\\
\end{align}
\] The premium for line \(i\) is \(u_i\partial\pi / \partial u_i\) in
the usual way. Euler implies the premium ``adds-up'' because the
portfolio is homogeneous.

The result can be seen in a couple of ways. From the first defintion of
\(\pi\) as an \(\int_0^\alpha\) and the three differentials noted above
we get \[
\begin{align}
\nabla_u\pi &= \int_0^{\alpha(X)} \nabla_u(g(S_u(t)))dt + g(S_u(\alpha))\nabla_u\alpha(u) \\
&=\int_0^{\alpha(X)} g'(S_u(t))\text{E}[X_i 1_{\{X=t\}}] dt + g(S_u(\alpha))E[X_i\mid X> F_u^{-1}(p) ] \\
&=\int_0^{\alpha(X)} g'(S_u(t))\text{E}[X_i \mid X=t] f_u(t) dt + g(S_u(\alpha))E[X_i\mid X> F_u^{-1}(p) ] \\
&=\text{E}\left[\left(X_i 1_{\{X\le \alpha\}} + E[X_i\mid X> F_u^{-1}(p)]1_{\{X > \alpha\}}\right) g'(S_u(X)) \right]
\end{align}
\] where the last line follows from the conditional expectation (tower
property) formula. Notice that the default expectation is take wrt
\(X>F_u^{-1}(p)\) whereas the outside integral is split at tvar.

Alternatively using the third definition of \(\pi\) \[
\begin{align}
\nabla_u\pi
&= \nabla_u \int_0^1 (F_u^{-1}(p)\wedge \alpha(u)) g'(1-p)dp  \\
&= \int_0^1 \left(\nabla_u F_u^{-1}(p) 1_{\{X_u\le \alpha\}} +
            \nabla_u \alpha(u) 1_{\{X_u > \alpha\}}\right) g'(1-p)dp  \\
&= \int_0^{F(\alpha)} \text{E}[X_i \mid X=F_u^{-1}(p) ] g'(1-p)dp + \text{E}[X_i \mid X> F_u^{-1}(p)] g(S(\alpha)) \\
&= \int_0^{\alpha} \text{E}[X_i \mid X=t ] g'(S(t))f(t)dt + \text{E}[X_i \mid X> F_u^{-1}(p)] g(S(\alpha)) \\
&= \text{E}\left[ \text{E}[X_i \mid X ]1_{\{X\le\alpha\}} g'(S(X_u)) \right]
+ \text{E}[X_i \mid X> F_u^{-1}(p)] g(S(\alpha)) \\
&= \text{E}\left[ X_i 1_{\{X\le\alpha\}} g'(S(X_u)) \right]
+ \text{E}[X_i \mid X> F_u^{-1}(p)] g(S(\alpha)) \\
\end{align}
\] It is important that \(g'(1-p)\) \emph{does not depend on \(u\)}. The
dependence on \(u\) has been subsumed into the re-ordering defined by
\(F_u^{-1}\). The fourth line follows from the third via substituting
\(F(t)=p\).

\hypertarget{properties-of-method-a}{%
\subsubsection{Properties of Method A}\label{properties-of-method-a}}

\begin{itemize}
\tightlist
\item
  \textbf{Assumes and relies critically on homogeneity}
\item
  Result is independent of the default rule
\item
  Produces core of fuzzy game and only measure consistent with
  performance management. (In fuzzy game approach there is an assumption
  that the underlying risks can be fractionally quota shared, not that
  they are homogeneous themselves.)
\end{itemize}

\hypertarget{method-b}{%
\subsection{Method B}\label{method-b}}

Portfolio \(X(u)\) that may or may not be homogeneous. An inhomogenous
example is \(X(u)=X_1(u_1) + X_2(u_2)\) where \(X_i(u_i)\) is a compound
Poisson distribution with expected claim count \(u_i\).

Regulatory capital \(\alpha\) determined by TVaR. Same distortion \(g\)
determines cost of risk transfer.

The insurer sells the residual value of the risk when supported by
assets of \(a\) to the capital markets for a premium \(P(a)\). The sale
is structured by tranche, so the premium is an integral of the premium
density. For a layer attaching at \(x\) (i.e.~a layer \(x\) to \(x+dx\))
the expected loss cost is \(S(x)dx\). As is well known, the expected
losses upto total capital \(a\) are given by \[
\int_0^a S(x)dx.
\] The distortion function \(g\) prices the layer at \(x\) as
\(g(S(x))\). Therefore the total premium is \[
P(a) = \int_0^a g(S(x))dx.
\] The premium density is therefore \[
P'(x) = \frac{dP}{da} = g(S(x)).
\]

The insurer now faces the problem of allocating the premium
\(P(\alpha(X))\) back to lines 1 and 2. Use a co-measure approach---this
is exavtly what our spreadsheet did in the CAS sessions! We have a
stable world, with \(u_1,u_2\) are fixed. The pricing distortion
function \(g\) combined with fixed \(X\) gives a probability distortion
\(g'(S(X))\), i.e.~and \(E_{\Bbb{Q}}[X] := E[Xg'(S(X))]\). Standard
finance theory says that in this situation, with a linear pricing rule,
we price any payoff \(Y\) by \$
E\_\{\Bbb{Q}\}{[}Y{]}=E{[}Yg'(S(X)){]}\$.

We have to specify the payoff to line \(i\) when the insurer only has
assets \(a\). Consider two possibilities.

\begin{enumerate}
\def\labelenumi{\arabic{enumi}.}
\tightlist
\item
  The two lines have equal priority in event of default. The payment to
  line \(i\) is \[
  X_i(a):=
  \begin{cases}
  X_i & X \le a \\
  X_i\frac{a}{X} & X > a \\
  \end{cases}
  = X_i \frac{X\wedge a}{X}.
  \]
\item
  Line 2 has lower priority than line 1. In this case the payout to line
  1 is \(X_1\wedge a\) and the payout to line 2 is \[
  X_2(a):=
  \begin{cases}
  X_2    & X \le a \\
  a-X_1  & X_1 \le a < X  \\
  0      & X_1 > a.
  \end{cases}
  \]
\end{enumerate}

In the case of equal priority the expected recovery to line \(i\) when
the insurer has assets \(a\) is \[
\begin{align}
\text{E}[X_i(a)] &= \text{E}[X_i \mid X \le a ]F(a) + a \text{E}\left[\frac{X_i}{X} \mid X > a \right] S(a) \\
&= \text{E}\left[ X_i\frac{X\wedge a}{X} \right].
\end{align}
\] From the second expression it is obvious this is an additive
allocation---in all cases, not just when \(X\) is a homogeneous family.

Finance theory now tells us the price of the contingent payout
\(X_i(\alpha)\) is \[
\begin{align}
P_i(u)=P_i(\alpha(X_u) &= \text{E}_{\Bbb{Q}}\left[  X_i\frac{X\wedge a}{X} \right] \\
&= \text{E}\left[  X_i\frac{X\wedge a}{X} g'(S(X)) \right] \\
&= \text{E}\left[ \text{E}\left[  X_i\frac{X\wedge a}{X} g'(S(X)) \mid X \right] \right] \\
&= \text{E}\left[ \text{E}\left[\frac{X_i}{X}\mid X \right] (X\wedge a) g'(S(X)) \right] \\
&= \text{E}\left[ \text{E}\left[ X_i \mid X \le \alpha \right] g'(S(X)) \right] (1-g(S(\alpha)))
    + a\text{E}\left[ \text{E}\left[\frac{X_i}{X}\mid X>\alpha \right] g'(S(X)) \right] g(S(\alpha)) \\
&= \text{E}\left[ X_i 1_{\{X \le \alpha \}} g'(S(X)) \right]
    + a\text{E}\left[ \frac{X_i}{X} g'(S(X)) 1_{\{ X>\alpha\}} \right] \\
&= \text{E}_{\Bbb{Q}}\left[ X_i 1_{\{X \le \alpha \}} + a \frac{X_i}{X} 1_{\{ X>\alpha\}} \right] \\
\end{align}
\]

The price component when the insurer remains solvent is the same as for
Method A. However the value of the payment made in the event of
insolvency is different---and depends on the explicit default
assumption: \[
\begin{align}
\text{Method A: } & \text{E}[X_i | X > \alpha]g(S(\alpha) \\
\text{Method B: } & a\text{E}\left[ \frac{X_i}{X} g'(S(X)) \mid X>\alpha \right] g(S(\alpha))
= a\text{E}\left[ \text{E}[\frac{X_i}{X} \mid X] g'(S(X)) 1_{\{ X>\alpha\}} \right] \\
\end{align}
\]

In method A the expectation is the allocation of total assets \(\alpha\)
to line \(i\).

Similar calculations can be performed for the case Line 2 has lower
priority.

\hypertarget{properties-of-method-b}{%
\subsubsection{Properties of Method B}\label{properties-of-method-b}}

\begin{itemize}
\tightlist
\item
  Does \textbf{not} assume or rely on homogeneity
\item
  Result dependents particular the default rule
\end{itemize}
